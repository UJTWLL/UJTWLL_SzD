\Chapter{Bevezetés}

A felsőoktatásban részt vevő hallgatók számos tantárgyat tanulmányoznak az itt eltöltött féléveik alatt. Ezeknek a tárgyaknak pedig szükségük van egy, a mai világban már elengedhetetlennek számító internetes felületre, amelyet a hallgatók bárhonnan, bármikor elérhetnek.

A szakdolgozatom célja az, hogy a GEIAK130B Mesterséges Intelligencia tárgyhoz egy ilyen felületet hozzon létre, webapplikáció formájában. Ezen a felületen elérhetővé válnak a tantárgyhoz kapcsolódó információk és anyagok, például a konzultációs időpontok, a zárthelyi dolgozat időpontjai, vagy az előadásfóliák.

Ügyelve arra, hogy csak illetékesek (egyetemi hallgatók) férjenek hozzá ehhez az oldalhoz, egy bejelentkezési felületre lesz szükségünk. Erről az oldalról csak akkor léphetnek tovább a felhasználók, ha rendelkeznek fiókkal. A regisztrált fiókok adatait pedig egy online adatbázisban tároljuk.

Tekintettel kell lennünk továbbá arra is, hogy szükségünk lesz egy olyan személyre is, aki felügyeli és karbantartja az alkalmazást, valamint a fiókokat. Ezt a feladatkört látja el majd a későbbiekben az adminisztrátor, akinek fő feladata a felhasználók regisztrálása, illetve adatbázisból való eltávolítása lesz.
