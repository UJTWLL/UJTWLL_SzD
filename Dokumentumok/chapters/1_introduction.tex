\Chapter{Bevezetés}

A felsőoktatásban részt vevő hallgatók számos tantárgyat hallgathatnak az itt eltöltött féléveik alatt. Ezeknek a tárgyaknak pedig szükségük van egy, a mai világban már elengedhetetlennek számító internetes felületre, amelyet a hallgatók bárhonnan, bármikor elérhetnek.

A szakdolgozatom célja az, hogy a GEIAK130B Mesterséges Intelligencia tárgyhoz egy ilyen felületet hozzon létre, webes alkalmazás formájában. Ezen a felületen elérhetővé válnak a tantárgyhoz kapcsolódó információk és anyagok, például a konzultációs időpontok, a zárthelyi dolgozat időpontjai, vagy az előadás során bemutatott prezentációk.

Ügyelve arra, hogy csak illetékesek (egyetemi hallgatók) férjenek hozzá ehhez az oldalhoz, egy bejelentkezési felületre lesz szükségünk. Erről az oldalról csak akkor léphetnek tovább a felhasználók, ha rendelkeznek fiókkal. A regisztrált fiókok adatait pedig egy online adatbázisban tároljuk.

Tekintettel kell lennünk továbbá arra is, hogy szükségünk lesz egy olyan személyre is, aki felügyeli és karbantartja az alkalmazást, valamint a fiókokat. Ezt a feladatkört látja el majd a későbbiekben az adminisztrátor, akinek fő feladata a felhasználók regisztrálása, illetve adatbázisból való eltávolítása lesz. 

Megjegyzendő továbbá, hogy a tárgyat oktató tanárnak képesnek kell lennie tananyagot hozzáadni, valamint eltávolítani, ezért hozzá kell férnie az alkalmazáshoz, ezáltal kezelni tudja a megjelenítendő dokumentumokat, illetve programokat.