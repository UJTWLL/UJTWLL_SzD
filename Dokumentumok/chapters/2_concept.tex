\Chapter{Tervezés}

\Section{Piackutatás}

Ezen feladat egészen szerteágazó tudást igényel, továbbiakban ezeket fejteném ki. Tekintettel arra, hogy lényegében egy weboldal az alapja az egésznek, így elengedhetetlen némi HTML ismeret, melyet – a középiskolai tanulmányaimon kívül – a már korábban említésre került Webtechnológiák I című tárgy keretein belül volt szerencsém elsajátítani. Ebben segítségemre lesz a tárgy előadója (Agárdi Anita), valamint az általa létrehozott GitHub repository \cite{webtechgithub}. Itt rendelkezésemre állnak a gyakorlati feladatok, valamint olyan példafeladatok, melyek segítségével könnyebben megértheti bárki a nyelv sajátosságait. Emellett, amennyiben ez mégsem lenne kielégítő, úgy a W3 konzorcium oldalán további segítség található \cite{w3schools}. Ezen weboldalon számtalan programozási nyelv számára - így természetesen a HTML számára is – elérhetőek ismeretterjesztő szövegek, demonstrációs programok.

Ha már webprogramozás a téma, nem maradhat említés nélkül a dinamikus weboldal szíve – a JavaScript. A korábban már említett HTML nyelv mellett a JavaScript-tel is foglalkoztunk a Webtechnológiák I. tantárgy keretein belül. Továbbá, szakmai gyakorlatom során is volt részem nem kevés JavaScript programozásban.

Egy weboldal harmadik, egyben utolsó összetevője a stílus, a kinézet. Ezt CSS stíluslapokkal tervezem megoldani. Ebből a nyelvből mind középiskolában, mind a már oly sokszor említésre került Webtechnológiák I. tárgy tanulmányozása során mélyíthettem tudásomat.

Viszont itt még nem állhatunk meg. Ahhoz, hogy ebből egy szerteágazó webalkalmazás legyen, elengedhetetlen valamilyen backend keretrendszer használata. A következő alfejezetben felsorakoztatunk ezek közül néhányat.

Adattárolási módszerekre is szükségünk lesz. Tekintettel a potenciális felhasználók körére, valamint a biztonsági elvárásokra, valamilyen online adatbázisra is szükségünk lesz. Ezek lehetőségeire szintén hozunk pár példát a későbbiekben.

\Section{Technológia}

A szakdolgozat feladat egy olyan rendszer létrehozását igényli, melybe a Miskolci Egyetem hallgatói – előzetes adminisztrátori regisztráció után – beléphetnek, így hozzáférve a Mesterséges Intelligencia című tantárgyhoz kapcsolódó anyagokhoz, példának okáért:

\begin{itemize}
\item az ütemtervhez,
\item az előadásanyagokhoz,
\item a zárthelyi dolgozat eredményeihez, vagy akár
\item a záróvizsga tételeihez.
\end{itemize}

Elérhetőek lennének a tárgy előadásain, valamint gyakorlatain bemutatott programok, például a BrainMaker vagy a betanítható sakkrobot. Továbbá, biztosítani kell egy feltöltésre alkalmas felületet, amennyiben a felhasználók saját tartalmat szeretnének rendelkezésre bocsátani. Természetesen ez nem jelenne meg azonnal a weboldalon, valószínűleg illetékes (például adminisztrátori) jóváhagyás után kerülne fel ténylegesen a rendszerbe. Mindezek mellett az adminisztrátor(ok) számára is biztosított lenne egy felület, ahol
\begin{itemize}
\item hozzáadhatnak,
\item módosíthatnak, vagy akár
\item törölhetnek
\end{itemize}felhasználói fiókokat.

\Section{Követelmény specifikáció}

Az igényes munka, a biztonság és a felhasználói élmény kívánalmai miatt követelményeket állítottunk fel magunknak a megvalósítani kívánt rendszerrel kapcsolatban.

Először is, a bejelentkezési rendszernek megfelelően biztonságosnak kell lennie. Ezt online adatbázisrendszerrel, titkosított jelszavakkal, valamint érvényesítési metódusokkal terveztük elérni.

A felhasználói kezelhetőség javítása érdekében pedig menürendszert, valamint stíluslapokat vezetünk be, ezáltal fejlesztve a rendszer megjelenését.

Természetesen ezeket az elvárásokat a felhasználói felület mellett az adminisztrációs felületre is alkalmazzuk.