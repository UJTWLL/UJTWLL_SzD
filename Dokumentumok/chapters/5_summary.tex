\Chapter{Összefoglalás}

A szakdolgozatom létrehozása során betekintést nyerhettem a manapság aktuális, webfejlesztéshez használatos eszközökhöz, így a Node.js, valamint a MongoDB rendszerekkel való munkavégzés által számottevő tudásra tettem szert. Ezeken belül megtanulhattam, hogy hogyan épül fel egy autentikációs folyamat, mely magában foglalja az adatbázissal való kommunikációt is. Emellett elsajátíthattam egy reszponzív webalkalmazás fejlesztését, a menürendszertől a fájlfeltöltés-kezeléseken át egészen a munkamenet-kezelésig. Alkalmam nyílt a szoftvertesztelés apróbb, ám annál hatásosabb és segítökészebb folyamataiba is.

A MongoDB keretein belül betekinthettem egy NoSQL adatbázisrendszer műkódésének folyamatába. Megtanultam e rendszernek adattárolási metódusait, műveleteit. Ezen felül képes lettem titkosított adatokkal dolgozni, mely például a felhasználói fiókok implementációjánál különösen hasznos (sőt, elvárható).

A szakdolgozatom készítése során segítségemre volt a Webtechnológiák I tantárgy során elsajátított tudás, illetve a szakmai gyakorlatom során, JavaScript nyelven végzett munkatapasztalat is.

Annak ellenére, hogy számos területen mélyíthettem tudásomat a szakdolgozat készítése során, természetesen van lehetőségem fejlődni ezen a területen. Például a visszajelzések és hibaüzenetek rendszerét meg lehetett volna oldani úgy, hogy ne kelljen új weboldalt megjeleníteni. Továbbá, a teljes rendszer single-page megoldásként is elkészíthető lett volna, viszont ehhez érdemesebb lett volna egy alkalmasabb keretrendszert (például Angular-t) használni. Illetve, több fájl egyidejűleg történő feltöltését is megemlíthejtük, mint fejlesztési opció.

Összességében, sikerült egy - nem tökéletes, de - kielégítő webalkalmazást létrehozni.